% TEMPLATE for Usenix papers, specifically to meet requirements of
%  USENIX '05
% originally a template for producing IEEE-format articles using LaTeX.
%   written by Matthew Ward, CS Department, Worcester Polytechnic Institute.
% adapted by David Beazley for his excellent SWIG paper in Proceedings,
%   Tcl 96
% turned into a smartass generic template by De Clarke, with thanks to
%   both the above pioneers
% use at your own risk.  Complaints to /dev/null.
% make it two column with no page numbering, default is 10 point

% Munged by Fred Douglis <douglis@research.att.com> 10/97 to separate
% the .sty file from the LaTeX source template, so that people can
% more easily include the .sty file into an existing document.  Also
% changed to more closely follow the style guidelines as represented
% by the Word sample file. 

% Note that since 2010, USENIX does not require endnotes. If you want
% foot of page notes, don't include the endnotes package in the 
% usepackage command, below.

%----------------------------------------------------------------------------------------
%	PACKAGES & CONFIG
%----------------------------------------------------------------------------------------

%\documentclass[paper=a4, fontsize=11pt]{article} % A4 paper and 11pt font size
\documentclass[letterpaper,twocolumn,10pt]{article}

\usepackage{usenix,epsfig,endnotes}
\usepackage[T1]{fontenc} % Use 8-bit encoding that has 256 glyphs
\usepackage[english]{babel} % English language/hyphenation
\usepackage{amsmath,amsfonts,amsthm} % Math packages
\usepackage{listings} % source code package
\usepackage{tikz} % graph package
\usepackage{supertabular} % table span multiple pages
\usetikzlibrary{shapes,arrows}
\usepackage{courier}
\usepackage[hyphens]{url} %for showing urls in bibliography
%\usepackage{hyperref} %for breaking urls in bib
%\hypersetup{colorlinks=true,breaklinks=true} %for showing urls

%\usepackage{sectsty} % Allows customizing section commands
%\allsectionsfont{\centering \normalfont\scshape} % Make all sections centered, the default font and small caps
%\setlength\parindent{0pt} % Removes all indentation from paragraphs

%----------------------------------------------------------------------------------------
%	TITLE
%----------------------------------------------------------------------------------------
\usepackage{titling}
\newcommand{\subtitle}[1]{%
  \posttitle{%
    \par\end{center}
    \begin{center}\large#1\end{center}
    \vskip0.5em}%
}

\begin{document}

%make title bold and 14 pt font (Latex default is non-bold, 16 pt)
\title{\Large \bf Title}
\subtitle{\textit{Subtitle - Project 2014}}


\author{
{\rm Gianluca Barbon}\\
gbarbon@dsi.unive.it
%\and
}

\date{July 24, 2014}
\maketitle

% Use the following at camera-ready time to suppress page numbers.
% Comment it out when you first submit the paper for review.
\thispagestyle{empty}


\subsection*{Abstract}
\emph{}

%----------------------------------------------------------------------------------------
%	1. Introduction
%----------------------------------------------------------------------------------------

\section{Introduction}
\paragraph{}

%----------------------------------------------------------------------------------------
%	2. First section
%----------------------------------------------------------------------------------------

\section{First section}
\paragraph{}

\subsection{First subsection}
\paragraph{}


%----------------------------------------------------------------------------------------
%	x. Conclusions
%----------------------------------------------------------------------------------------

\section{Conclusions}
\paragraph{}

\subsection{Future developments}
\paragraph{}
\cite{example} 

%----------------------------------------------------------------------------------------
%	NOTES AND BIBLIOGRAPHY
%----------------------------------------------------------------------------------------

{\footnotesize \bibliographystyle{acm}
\bibliography{biblio}
}

%\theendnotes

%column break
\vfill
\break

%----------------------------------------------------------------------------------------
%	APPENDICES
%----------------------------------------------------------------------------------------

\onecolumn
\appendix
\label{app:appendixA}
\lstset{language=Java}  
\section{Appendix: example code}

\subsection{Class test.java}


\label{app:appendixB}
\section{Appendix:}


%----------------------------------------------------------------------------------------
%	END OF DOCUMENT
%----------------------------------------------------------------------------------------

\end{document}







